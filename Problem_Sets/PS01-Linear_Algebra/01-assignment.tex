\documentclass[12pt]{article}
\usepackage{../../includes/lecture_notes}
\usepackage{../../includes/math}
\usepackage{../../includes/uark_colors}

\hypersetup{
  colorlinks = true,
  allcolors = ozark_mountains,
  breaklinks = true,
  bookmarksopen = true
}

\begin{document}
\begin{center}
  {\Huge\bf Topic \#1 Assignment}
  
  \smallskip
  {\large\it  ECON 5753 — University of Arkansas}

  \medskip
  {\large Prof. Kyle Butts}
\end{center}

These assignments should be completed in groups of 1 or 2 but submitted individually. 

\section*{Theoretical Questions}

\begin{enumerate}
  \item Let 
  $$
    \bm{A} = \begin{bmatrix} 6 & 7 & 9 \\ 1 & 2 & 3 \\ 8 & 4 & 6 \end{bmatrix}, 
    \quad 
    \bm{B} = \begin{bmatrix} 10 & 9 & 8 \\ 7 & 5 & 4 \\ 1 & 7 & 6 \end{bmatrix},
    \quad \text{ and } 
    \bm{C} = \begin{bmatrix} 1 & 0 & 2 \\ 0 & 1 & 0 \\ 4 & 0 & 2 \end{bmatrix}.
  $$ 
  Calculate the following:
  
  \begin{enumerate}
    \item $3\bm{B}^T + \bm{A}$
    \item $\bm{C}^T - 4\bm{A}$
    \item $(\bm{C}\bm{A})'$
  \end{enumerate}
  
  \bigskip\bigskip
  \item Let 
  $$
    \bm{X} = \begin{bmatrix} x_{11} & x_{12} & x_{13} \\ x_{21} & x_{22} & x_{23} \\ x_{31} & x_{32} & x_{33} \end{bmatrix}, 
    \quad 
    \beta = \begin{bmatrix} \beta_1 \\ \beta_2 \\ \beta_3 \end{bmatrix}.
  $$ 
  Verify the following two are equivalent:

  $$
    X \beta \quad \text{ and } 
    X_{\cdot, 1} \beta_1 + X_{\cdot, 2} \beta_2 + X_{\cdot, 3} \beta_3 
  $$

  
  \newpage
  \item Match each matrix with its inverse:
  
  $$
    A = \begin{bmatrix} 1 & 2 & -1 \\ -2 & 0 & 1 \\ 1 & -1 & 0 \end{bmatrix}, 
    \quad 
    B = \begin{bmatrix} 2 & 4 & 1 \\ -1 & 1 & -1 \\ 1 & 4 & 0 \end{bmatrix}, 
    \quad \text{ and } 
    C = \begin{bmatrix}  1 & 2 & 3 \\ 2 & 4 & 0 \\ 0 & 0 & 3 \end{bmatrix}
  $$
  
  $$
    a = \begin{bmatrix} -4 & -4 & 5 \\ 1 & 1 & -1 \\ 5 & 4 & -6 \end{bmatrix}, 
    \quad 
    b = \begin{bmatrix} 1 & 1 & 2 \\ 1 & 1 & 1 \\ 2 & 3 & 4\end{bmatrix}, 
    \quad \text{ and } 
    c = \text{ Not invertible}.
  $$


  \bigskip\bigskip
  \item Let $$
    \varepsilon = \begin{bmatrix}\varepsilon_1 \\ \varepsilon_2\end{bmatrix} \sim \mathcal{N}\left( \begin{bmatrix}1 \\ 1\end{bmatrix}, \begin{bmatrix}\sigma_1^2 & \sigma_{12} \\ \sigma_{12} & \sigma_2^2\end{bmatrix} \right).
  $$
  
  What is the distribution of $\begin{bmatrix} 1 & 1 \end{bmatrix} \varepsilon$?

  \bigskip\bigskip
  \item Let $f(x, y) = x^2 + y^2 + 2xy$.
  \begin{enumerate}
    \item What is $\frac{\partial}{\partial x} f(x,y)$ and $\frac{\partial}{\partial y} f(x,y)$?
    
    \item Use this to create a taylor approximation of this function at $(x_0, y_0)$. Write this in the form of:
    $$
      f(x_0 + dx, y_0 + dy) \approx f(x_0, y_0) + \begin{bmatrix}\frac{\partial}{\partial x} f(x_0,y_0) & \frac{\partial}{\partial x} f(x_0,y_0)\end{bmatrix} \begin{bmatrix}dx \\ dy\end{bmatrix}
    $$

    \item Plug in $(x_0, y_0) = (0, 0)$ to create a linear approximation to this function.
  \end{enumerate}
\end{enumerate}


\end{document}
